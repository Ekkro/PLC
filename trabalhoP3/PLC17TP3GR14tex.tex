% !TeX spellcheck = <none>
\documentclass{report}
\usepackage{graphicx}
\usepackage[portuguese]{babel}
\usepackage[utf8]{inputenc}
\usepackage{hyperref}
\usepackage{indentfirst}
%\usepackage[latin1]{inputenc}
%\usepackage{url}
\usepackage{color}
\usepackage{enumerate}
\usepackage{alltt}
\usepackage{fancyvrb}
\usepackage{listings}
\usepackage{amsmath}
\newcommand{\keyword}[1]{\textsf{#1}}

\DefineVerbatimEnvironment{code}{Verbatim}{fontsize=\footnotesize}
%LISTING - GENERAL
\lstset{
	basicstyle=\small,
	numbers=left,
	numberstyle=\tiny,
	numbersep=5pt,
	breaklines=true,
    frame=tB,
	mathescape=true,
	escapeinside={(*@}{@*)}
	}


\title{Processamento de Linguagens e Compiladores\\ (3º ano de LCC)\\ \textbf{Galo}\\ TP3\\ Grupo 14}
\author{Artur Queiroz\\ A77136 \and  Rafael Fernandes\\ A78242 \and Rafaela Pinho\\ A77293 }
\date{\today}

\begin{document}
	
\maketitle
	

\begin{abstract}
	Neste relatório apresentamos a linguagem que criamos e o complidor que gera o código para a Máquina Virtual VM.
\end{abstract}

\tableofcontents

\chapter{Introdução} \label{intro}
\indent

 
\chapter{Galo e Compilador} \label{fi}
\section{Descrição informal do problema}
\indent
Neste trabalho foi pedido para criarmos uma linguagem de programação imperativa e desenvolver um compilador para a linguagem criada.\\
\indent
Na linguagem as declarações de variáveis devem ser colocadas no início do programa, não pode haver re-declarações e não se pode usar variáveis sem estar declaradas primeiro. Caso não seja atribuido uma valor à variável depois da declaração, esta ficará com o valor zero.\\
\indent  	
O complilador deve gerar o código assembly para a Máquina Virtual VM.

\section{Especificação dos requisitos}
\indent
Para este tarbalho a linguagem que criamos tem de conter os seguintes requisitos:
\begin{enumerate}[1)]
	\item Declarar e manusear variáveis atómicas do tipo inteiro e estruturas do tipo array de inteiros.\\
	\item Ler do standard input e escrever no standard output.\\
	\item Fazer instruções básicas como a atribuição de expressões a variáveis.\\
	\item definir e invocar subprogramas sem parâmetros mas que possam retornar um resultado atómico(?)
	\item efetuar instruções para controlo do fluxo de execução—condicional e cíclica—que possam ser aninhadas. (?)
\end{enumerate}
\section{Expressões regulares} 
As expressões regulares usadas foram:

\begin{enumerate}[1)]
	\item

\end{enumerate}

\section{A nossa linguagem}
\subsection{Galo}
\indent
Como já referido em cima, foi-nos pedidos para criar uma linguagem de programação. Decidimos chamar de Galo por ser um símbolo típico de Potugal, e atribuimos .gl para a extensão.\\
\indent
Para a definirmos utilizamos uma gramática independente do contexto, em que tomamos certas decisões que serão especificadas.\\
\indent
O Galo reconhece os segintes tipos: números inteiros (int), números décimais  (float) e conjunto de caractéres (string).


\begin{code}
	

\end{code}


\chapter{Codificação e Testes}
\section{Problemas de implementação, Decisões e Alternativas}
\subsection{Problemas de implementação}

   


\subsection{Decisões}
\begin{enumerate}[1)]
	\item O E ($\&\&$) está definida pela multuplicação e o OU ($||$) pela adição.
 
	 \begin{table}[h]
		\begin{center}
	 	\caption{Tabela do E}
	 	\begin{tabular}{r|lr}
	 	 *(E)& 0 & 1\\ % Note a separação de col. e a quebra de linhas
	 	\hline          % para uma linha horizontal
	 	0 & 0 & 0 \\
	 	1 & 0 & 1\\
 	
		\end{tabular}
		\caption{Tabela do OU}
		\begin{tabular}{r|lr}
		+(OU)& 0 & 1\\ % Note a separação de col. e a quebra de linhas
		\hline          % para uma linha horizontal
		0 & 0 & 1 \\
		1 & 1 & 2\\
		\end{tabular}
		\end{center}
	\end{table}

	\item Não se pode declarar mais do que uma variável numa linha.\\
	 Exemplo: \\int a = 2 , c = 0;\\ terá de ser:\\
	 int a = 2;\\
	 int c = 0;\\
	 \item Não se pode fazer "return" dentro dos Se's e dos Enq's.\\ 
	 \item Nas expressões numéricas, as operações binárias têm de estar sempre dentro de parênteses.
	 \item O Se tem de ter sempre Senao.
	
\end{enumerate}


\subsection{Alternativas}



\section{Testes realizados e Resultados}





\chapter{Conclusão} \label{concl}

Hoje em dia já existem muitos pré-processadores que facilitam a escrita de documentos em HTML.\\
Tendo em conta os aspetos apresentados ao longo do relatório, conclui-se que o $Flex$ é uma boa ferramenta para fazer o pré-processador e que com ele se torna fácil programar usando expressões regulares e um pouco de linguagem C. \\
Como trabalho futuro poderíamos acrescentar mais símbolos para completar o pré processador.




\end{document}
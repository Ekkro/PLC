%
% Layout retirado de http://www.di.uminho.pt/~prh/curplc09.html#notas
%
\documentclass{report}
\usepackage[portuges]{babel}
%\usepackage[utf8]{inputenc}
\usepackage[latin1]{inputenc}

\usepackage{url}
\usepackage{enumerate}

%\usepackage{alltt}
%\usepackage{fancyvrb}
\usepackage{listings}
%LISTING - GENERAL
\lstset{
	basicstyle=\small,
	numbers=left,
	numberstyle=\tiny,
	numbersep=5pt,
	breaklines=true,
    frame=tB,
	mathescape=true,  
	escapeinside={(*@}{@*)}
}
%
%\lstset{ %
%	language=Java,							% choose the language of the code
%	basicstyle=\ttfamily\footnotesize,		% the size of the fonts that are used for the code
%	keywordstyle=\bfseries,					% set the keyword style
%	%numbers=left,							% where to put the line-numbers
%	numberstyle=\scriptsize,				% the size of the fonts that are used for the line-numbers
%	stepnumber=2,							% the step between two line-numbers. If it's 1 each line
%											% will be numbered
%	numbersep=5pt,							% how far the line-numbers are from the code
%	backgroundcolor=\color{white},			% choose the background color. You must add \usepackage{color}
%	showspaces=false,						% show spaces adding particular underscores
%	showstringspaces=false,					% underline spaces within strings
%	showtabs=false,							% show tabs within strings adding particular underscores
%	frame=none,								% adds a frame around the code
%	%abovecaptionskip=-.8em,
%	%belowcaptionskip=.7em,
%	tabsize=2,								% sets default tabsize to 2 spaces
%	captionpos=b,							% sets the caption-position to bottom
%	breaklines=true,						% sets automatic line breaking
%	breakatwhitespace=false,				% sets if automatic breaks should only happen at whitespace
%	title=\lstname,							% show the filename of files included with \lstinputlisting;
%											% also try caption instead of title
%	escapeinside={\%*}{*)},					% if you want to add a comment within your code
%	morekeywords={*,...}					% if you want to add more keywords to the set
%}

\usepackage{xspace}

\parindent=0pt
\parskip=2pt

\setlength{\oddsidemargin}{-1cm}
\setlength{\textwidth}{18cm}
\setlength{\headsep}{-1cm}
\setlength{\textheight}{23cm}

\def\darius{\textsf{Darius}\xspace}
\def\antlr{\texttt{AnTLR}\xspace}
\def\pe{\emph{Publicação Eletrónica}\xspace}

\def\titulo#1{\section{#1}}
\def\super#1{{\em Supervisor: #1}\\ }
\def\area#1{{\em \'{A}rea: #1}\\[0.2cm]}
\def\resumo{\underline{Resumo}:\\ }


%%%%\input{LPgeneralDefintions}

\title{Processamento de Linguagens e Compiladores 3º ano\\ \textbf{Expressões Regulares e GAWK}\\ Relatório de Desenvolvimento do grupo 14}
\author{Artur Queiroz\\ (A77136) \and \\ Rafael Fernandes\\ (A78242) \and \\ Rafaela Pinho\\ A77293  }
\date{\today}

\begin{document}

\maketitle

\begin{abstract}
Este trabalho foca-se os conceitos básicos do funcionamento GAWK e das expressões regulares utilizadas para descrever padrões.
Neste relatório descrevemos as decisões tomadas e as dificuldades encontradas bem como pequenos exemplos, para que qualquer um que o leia perceba facilmente como funciona o nosso projeto.
\end{abstract}

\tableofcontents

\chapter{Introdução} \label{intro}
Uma parte importante do sistema linux são as expressões regulares, a capacidade de procurar num input um determinado padrão facilita muitas das operaçoẽs.
\titulo{Um belo Projeto}
\area{Processamento de Linguagens}
blablabla
\super{José João}
\titulo{Um belo Dia em Briteiros}


\begin{description}
  \item [Enquadramento] \textbf{bla bla} bla bla
  \item [Conteúdo do documento] \textsf{ble ble} \texttt{ble} ble
  \item [Resultados -- pontos a evidenciar] \textit{bli bli bli bli}
  \item [Estrutura do documento] \underline{blo blo blo}
\end{description}

letras gregas são estas $ \alpha \beta \gamma \delta $ que aqui demonstro

exemplo simples de fração \[ \frac{\frac{a * b + c}{4-3}}{3*5} \] simples

Mais exemplos de listas enumeradas mas agora com letras:
\begin{enumerate}[a)]
\item Listar todas as Pessoas identificadas, sem repetições;
\item Listar os Países e Cidades marcadas;
\item Listar as Organizações.
\end{enumerate}

A mesma enumeração mas no standard numérico
\begin{description}
\item[Etape 1:] Listar todas as Pessoas identificadas, sem repetições;
\item[Etape 2:] Listar os Países e Cidades marcadas;
\item[Etape 3:] Listar as Organizações.
\end{description}

\section*{Estrutura do Relatório} \
explicar como está organizado o documento, referindo os capítulos existentes em~\cite{Pereira201635}
e a sua articulação explicando o conteúdo de cada um.
No capítulo~\ref{fi} faz-se uma análise detalhada do problema proposto
de modo a poder-se especificar  as entradas, resultados e formas de transformação.\\
etc. \ldots\\
No capítulo~\ref{concl} termina-se o relatório com uma síntese do que foi dito,
as conclusões e o trabalho futuro

\chapter{Ficheiros de internacionalização} \label{fi}

\section{Descrição informal do problema}\
É necessária um "script" que permita:
\begin{enumerate}[a)]
\item Analisar vários ficheiros de formato PO, e devolver o número de traduções e os seus tradutores, bem como alguns metadados acharmos relevantes.
\item Através de ficheiros PO's criar dicionários de triangulação de Português para Francês.
\item Reformatar ficheiros de formato PO pra LaTeX.
\end{enumerate}

\section{Especificação do Requisitos}
Os requisitos mínimos deste tarbalho são saber utilizar Linux, bash e GAWK. 

\chapter{Concepção/desenho da Resolução}
\section{Estruturas de Dados}
\section{Algoritmos}
Utilizamos as seguintes expressões regulares:
\begin{enumerate}[i)]
\item /^msgid/ \rightarrow "msgid" 
\item /^msgid[ \t]*""/ \rightarrow "msgid " (com tab)
\item /^"Language:/ \rightarrow
\item /<.*@.*>/ && !/^"Report/ && !/FIRST/ && !/^"Language/ \rightarrow
\item /^"Language-Team: *\\n"/ \rightarrow
\item /^"Language-Team:/ \rightarrow
\item /portugu[êe]se?s?/ \rightarrow
\item /^msgid[ \t]*""/ \rightarrow
\item /^msgstr[ \t]*""/ \rightarrow
\item /^msgid *"/ \rightarrow
\item /^msgstr *"/ \rightarrow





\chapter{Codificação e Testes}
\section{Alternativas, Decisões e Problemas de Implementação}
\section{Testes realizados e Resultados}
Mostram-se a seguir alguns testes feitos (valores introduzidos) e
os respectivos resultados obtidos:

%\VerbatimInput{teste1.txt}


\chapter{Conclusão} \label{concl}
Síntese do Documento~\cite{Martini2016a,Hoare73a}.\\
Estado final do projecto; Análise crítica dos resultados~\cite{Sto77a}.\\
Trabalho futuro.

\appendix
\chapter{Código do Programa}

Lista-se a seguir o código \antlr~\cite{Par05} do programa \darius~\cite{NPH2016a} que foi desenvolvido.
\begin{verbatim}
public class Aula()
  {
    int n, m;
    int max(int a, int b)
      {
       ......
       return(max);
      }
  }
\end{verbatim}

\begin{verbatim}
llll sanjdb c kjnjcnjnjj mmmmmmmmmmmmm hhhhhhhhhhhhhhhhhhhhhhhh jjjjjjjjjjjjjjjjjjjjjjjjjjjj kkkkkkkkkkkkkkkkkk
      aqui deve aparecer o código do programa,
      tal como está formato no ficheiro-fonte "darius.java"
      caso indesejável $\varepsilon$
\end{verbatim}

\begin{lstlisting}[caption={Exemplo de uma Listagem}, label={lstExe1}]
llll sanjdb c kjnjcnjnjj mmmmmmmmmmmmm hhhhhhhhhhhhhhhhhhhhhhhh jjjjjjjjjjjjjjjjjjjjjjjjjjjj kkkkkkkkkkkkkkkkkk
      ou então aparecer aqui neste sítio um pouco de matematica $\$$
      como alternativa ao anterior.
      e aqui mais um teste $\varepsilon$
\end{lstlisting}

É ainda possível importar diretamente o ficheiro:
\lstinputlisting{plfiltrobase.l} %input de um ficheiro

\bibliographystyle{alpha}
\bibliography{relprojLayout}















\end{document} 
%
% Layout retirado de http://www.di.uminho.pt/~prh/curplc09.html#notas
%
\documentclass{report}
\usepackage[portuguese]{babel}
\usepackage[utf8]{inputenc}
\usepackage[latin1]{inputenc}

\usepackage{url}
\usepackage{enumerate}

\usepackage{alltt}
\usepackage{fancyvrb}
\usepackage{listings}
%LISTING - GENERAL
\lstset{
	basicstyle=\small,
	numbers=left,
	numberstyle=\tiny,
	numbersep=5pt,
	breaklines=true,
    frame=tB,
	mathescape=true,  
	escapeinside={(*@}{@*)}
}

%\lstset{ %
%	language=Java,							% choose the language of the code
%	basicstyle=\ttfamily\footnotesize,		% the size of the fonts that are used for the code
%	keywordstyle=\bfseries,					% set the keyword style
%	%numbers=left,							% where to put the line-numbers
%	numberstyle=\scriptsize,				% the size of the fonts that are used for the line-numbers
%	stepnumber=2,							% the step between two line-numbers. If it's 1 each line
%											% will be numbered
%	numbersep=5pt,							% how far the line-numbers are from the code
%	backgroundcolor=\color{white},			% choose the background color. You must add \usepackage{color}
%	showspaces=false,						% show spaces adding particular underscores
%	showstringspaces=false,					% underline spaces within strings
%	showtabs=false,							% show tabs within strings adding particular underscores
%	frame=none,								% adds a frame around the code
%	%abovecaptionskip=-.8em,
%	%belowcaptionskip=.7em,
%	tabsize=2,								% sets default tabsize to 2 spaces
%	captionpos=b,							% sets the caption-position to bottom
%	breaklines=true,						% sets automatic line breaking
%	breakatwhitespace=false,				% sets if automatic breaks should only happen at whitespace
%	title=\lstname,							% show the filename of files included with \lstinputlisting;
%											% also try caption instead of title
%	escapeinside={\%*}{*)},					% if you want to add a comment within your code
%	morekeywords={*,...}					% if you want to add more keywords to the set
%}

\usepackage{xspace}

\parindent=0pt
\parskip=2pt

\setlength{\oddsidemargin}{-1cm}
\setlength{\textwidth}{18cm}
\setlength{\headsep}{-1cm}
\setlength{\textheight}{23cm}

\def\darius{\textsf{Darius}\xspace}
\def\antlr{\texttt{AnTLR}\xspace}
\def\pe{\emph{Publicação Eletrónica}\xspace}

\def\titulo#1{\section{#1}}
\def\super#1{{\em Supervisor: #1}\\ }
\def\area#1{{\em \'{A}rea: #1}\\[0.2cm]}
\def\resumo{\underline{Resumo}:\\ }


%%%%\input{LPgeneralDefintions}

\title{Processamento de Linguagens e Compiladores 3º ano\\ \textbf{Expressões Regulares e GAWK}\\ Relatório de Desenvolvimento do grupo 14}
\author{Artur Queiroz\\ (A77136) \and \\ Rafael Fernandes\\ (A78242) \and \\ Rafaela Pinho\\ A77293  }
\date{\today}

\begin{document}



\begin{abstract}
Este trabalho foca-se os conceitos básicos do funcionamento GAWK e das expressões regulares utilizadas para descrever padrões.
Neste relatório descrevemos as decisões tomadas e as dificuldades encontradas bem como pequenos exemplos, para que qualquer um que o leia perceba facilmente como funciona o nosso projeto.
\end{abstract}

\tableofcontents

\chapter{Introdução} \label{intro}
Uma parte importante do sistema linux são as expressões regulares, a capacidade de procurar num input um determinado padrão facilita muitas das operaçoẽs.
\titulo{Um belo Projeto}
\area{Processamento de Linguagens}
blablabla
\super{José João}
\titulo{Um belo Dia em Briteiros}


\begin{description}
  \item [Enquadramento] \textbf{bla bla} bla bla
  \item [Conteúdo do documento] \textsf{ble ble} \texttt{ble} ble
  \item [Resultados -- pontos a evidenciar] \textit{bli bli bli bli}
  \item [Estrutura do documento] \underline{blo blo blo}
\end{description}

letras gregas são estas $ \alpha \beta \gamma \delta $ que aqui demonstro

exemplo simples de fração \[ \frac{\frac{a * b + c}{4-3}}{3*5} \] simples

Mais exemplos de listas enumeradas mas agora com letras:
\begin{enumerate}[a)]
\item Listar todas as Pessoas identificadas, sem repetições;
\item Listar os Países e Cidades marcadas;
\item Listar as Organizações.
\end{enumerate}

A mesma enumeração mas no standard numérico
\begin{description}
\item[Etape 1:] Listar todas as Pessoas identificadas, sem repetições;
\item[Etape 2:] Listar os Países e Cidades marcadas;
\item[Etape 3:] Listar as Organizações.
\end{description}

\section*{Estrutura do Relatório} \
explicar como está organizado o documento, referindo os capítulos existentes em~\cite{Pereira201635}
e a sua articulação explicando o conteúdo de cada um.
No capítulo~\ref{fi} faz-se uma análise detalhada do problema proposto
de modo a poder-se especificar  as entradas, resultados e formas de transformação.\\
etc. \ldots\\
No capítulo~\ref{concl} termina-se o relatório com uma síntese do que foi dito,
as conclusões e o trabalho futuro

\chapter{Ficheiros de internacionalização} \label{fi}

\section{Descrição informal do problema}\
É necessária um "script" que permita:
\begin{enumerate}[a)]
\item Analisar vários ficheiros de formato PO, e devolver o número de traduções e os seus tradutores, bem como alguns metadados acharmos relevantes.
\item Através de ficheiros PO's criar dicionários de triangulação de Português para Francês.
\item Reformatar ficheiros de formato PO pra LaTeX.
\end{enumerate}

\section{Especificação do Requisitos}
Os requisitos mínimos deste tarbalho são saber utilizar Linux, bash e GAWK. 

\chapter{Concepção/desenho da Resolução}
\section{Estruturas de Dados}
\section{Algoritmos}
Utilizamos as seguintes expressões regulares:
\begin{enumerate}[i)]
\item /^msgid/  
\item /^msgid[ \t]*""/ 
\item /^"Language:/ 
\item /<.*@.*>/ && !/^"Report/ && !/FIRST/ && !/^"Language/ 
\item /^"Language-Team: *\\n"/ 
\item /^"Language-Team:/ 
\item /portugu[êe]se?s?/ 
\item /^msgid[ \t]*""/ 
\item /^msgstr[ \t]*""/ 
\item /^msgid *"/ 
\item /^msgstr *"/ 





\chapter{Codificação e Testes}
\section{Problemas de implementação, Decisões e Alternativas}
\titulo{Problemas de implementação}
Em geral, no trabalho encontramos problemas com a formatação de alguns ficheiros,pois não estavam na formatação prevista, e tivemos alguns problemas com o LaTex.
\titulo{Decisões}
No problema 1 proposto no exercício, decidimos ignorar os "msgid" que não contivessem nenhuma mensagem ou que contivessem mais de uma tradução.Decidimos ignorar os "msgid" que não contivessem nenhuma mensagem ou que contivessem mais de uma tradução.Escolhemos como metadados mais importantes a linguagem, a percentagem de blocos, quantos blocos tem o ficheiro e quantos e quem são os tradutores. Decidimos também que a linguagem só está no "Language".
No problema 2 decidimos utilizar a "Language-team" para definir a linguagem de tradução, pois nenhum dos ficheiros tinha linguagem no campo"Language". Usamos uma matriz descritiva sendo as linhas a frase em inglês e os descritivos das colunas a linguagem portuguesa ou francesa. Assumimos só a linguagem portuguesa e a francesa e imprimimos unicamente o que tem tradução en triangulação.
No último problema ignoramos o "\n" e o "#", devido ao LaTex, sustituindo por "" e por --, respetivamente.
\titulo{Alternativas}
Uma alternativa era modificar os ficheiros manualmente de modo a estar no formato certo, outra era fazer inúmeros casos especiais.   







\section{Testes realizados e Resultados}
Mostram-se a seguir alguns testes feitos (valores introduzidos) e
os respectivos resultados obtidos:

%\VerbatimInput{teste1.txt}


\chapter{Conclusão} \label{concl}
Síntese do Documento~\cite{Martini2016a,Hoare73a}.\\
Estado final do projecto; Análise crítica dos resultados~\cite{Sto77a}.\\
Trabalho futuro.

\appendix
\chapter{Código do Programa}

Lista-se a seguir o código \antlr~\cite{Par05} do programa \darius~\cite{NPH2016a} que foi desenvolvido.
\begin{verbatim}
public class Aula()
  {
    int n, m;
    int max(int a, int b)
      {
       ......
       return(max);
      }
  }
\end{verbatim}

\begin{verbatim}
llll sanjdb c kjnjcnjnjj mmmmmmmmmmmmm hhhhhhhhhhhhhhhhhhhhhhhh jjjjjjjjjjjjjjjjjjjjjjjjjjjj kkkkkkkkkkkkkkkkkk
      aqui deve aparecer o código do programa,
      tal como está formato no ficheiro-fonte "darius.java"
      caso indesejável $\varepsilon$
\end{verbatim}

\begin{lstlisting}[caption={Exemplo de uma Listagem}, label={lstExe1}]
llll sanjdb c kjnjcnjnjj mmmmmmmmmmmmm hhhhhhhhhhhhhhhhhhhhhhhh jjjjjjjjjjjjjjjjjjjjjjjjjjjj kkkkkkkkkkkkkkkkkk
      ou então aparecer aqui neste sítio um pouco de matematica $\$$
      como alternativa ao anterior.
      e aqui mais um teste $\varepsilon$
\end{lstlisting}

É ainda possível importar diretamente o ficheiro:
\lstinputlisting{plfiltrobase.l} %input de um ficheiro

\bibliographystyle{alpha}
\bibliography{relprojLayout}















\end{document} 